\documentclass{article}

\usepackage{amssymb}
\usepackage{bussproofs}
\usepackage{amsmath}
\usepackage{pdftexcmds}
\usepackage{fullpage}

% Blah blah
\DeclareFontFamily{U}{cbgreek}{}
\DeclareFontShape{U}{cbgreek}{m}{n}{
        <-6>    grmn0500
        <6-7>   grmn0600
        <7-8>   grmn0700
        <8-9>   grmn0800
        <9-10>  grmn0900
        <10-12> grmn1000
        <12-17> grmn1200
        <17->   grmn1728
      }{}
\DeclareFontShape{U}{cbgreek}{bx}{n}{
        <-6>    grxn0500
        <6-7>   grxn0600
        <7-8>   grxn0700
        <8-9>   grxn0800
        <9-10>  grxn0900
        <10-12> grxn1000
        <12-17> grxn1200
        <17->   grxn1728
      }{}

\DeclareRobustCommand{\qoppa}{%
  \text{\usefont{U}{cbgreek}{\normalorbold}{n}\symbol{19}}%
}
\makeatletter
\newcommand{\normalorbold}{%
  \ifnum\pdf@strcmp{\math@version}{bold}=\z@ bx\else m\fi
}
\makeatother
% End Blah

\newcommand{\MLTT}{\ensuremath{\mathsf{MLTT}}}
\newcommand{\CT}{\ensuremath{\mathsf{CT}}}
\newcommand{\CTTT}{\ensuremath{\mathsf{ChTT}}}

\newcommand{\Trm}[2]{\mathsf{Tm}[{{#1} \vdash {#2}}]}
\newcommand{\TNf}[2]{ {\mathsf{Nf}}[{{#1} \vdash {#2}}]}
\newcommand{\TEq}[3]{ {\mathsf{Cv}}[{{#1} \vdash {#2} \equiv {#3}}]}

\newcommand{\Code}{\Lambda}
\newcommand{\Quote}{\chi}
\newcommand{\Eval}{\qoppa}
\newcommand{\Red}{\mathrel{\downarrow}}
\newcommand{\Nat}{\mathbb{N}}
\newcommand{\TZero}{\mathtt{O}}
\newcommand{\TSucc}{\mathtt{S}}
\newcommand{\Type}{\square}

\newcommand{\QNf}[1]{{\ulcorner{#1}\urcorner}}

\newcommand{\CApp}{\bullet}
\newcommand{\CZero}{\mathbf{O}}
\newcommand{\CSucc}{\mathbf{S}}

\newcommand{\Reif}[2]{{[#1]}^{#2}}

\newcommand{\PNf}[1]{\mathsf{Nf}_{#1}}
\newcommand{\PNe}[1]{\mathsf{Ne}_{#1}}

\begin{document}

\section{Conventions}

We are going to manipulate three different logical levels:
\begin{itemize}
 \item The meta-theory, an ambient Agda-like constructive type theory not further specified
 \item The object theory, an extension of $\MLTT$ called $\CTTT$ and described in Section~\ref{CTTT}.
 \item Codes, closed terms of $\CTTT$ of a specific first-order type of codes called $\Code$.
\end{itemize}

Since this is confusing enough, we are going to be extremely careful about the precise level we are talking about. We will call \emph{internal} everything stated from within $\CTTT$ and \emph{external} those statements living in the meta-theory, and use notations to highlight the difference of levels. In Section~\ref{CTTT} we present the rules in the usual liberal way, but in-text definitions follow the following conventions. Statements in the object theory are written with brackets $[\cdot]$, e.g.
$\Trm{\Gamma}{A}$ is the meta-type of well-typed object terms of object type $A$ in object context $\Gamma$. We use a similar notation for hereditarily normal terms $\TNf{\Gamma}{A}$ and conversion $\TEq{\Gamma}{t}{u}$

\section{Object Type Theory}\label{CTTT}

We consider $\CTTT$, a one-universe $\MLTT$ enriched with the following data.\bigskip

\begin{center}
\renewcommand{\arraystretch}{2}
\begin{tabular}{cc}

\AxiomC{}
\UnaryInfC{$\vdash \Code : \Type$}
\DisplayProof
&
\AxiomC{}
\UnaryInfC{$\vdash {\Red} : \Code \rightarrow \Code \rightarrow \Type$}
\DisplayProof

\\

\AxiomC{}
\UnaryInfC{$\vdash \CApp : \Code \rightarrow \Code \rightarrow \Code$}
\DisplayProof

\\

\AxiomC{}
\UnaryInfC{$\vdash \CZero : \Code$}
\DisplayProof

&

\AxiomC{}
\UnaryInfC{$\vdash \CSucc : \Code \rightarrow \Code$}
\DisplayProof

\\

\AxiomC{}
\UnaryInfC{$\vdash \Quote : (\Nat \rightarrow \Nat) \rightarrow \Code$}
\DisplayProof
&
\AxiomC{$\Gamma \vdash f : \Nat \rightarrow \Nat$}
\AxiomC{$f \in \PNf{(\cdot)}$}
\BinaryInfC{$\Gamma \vdash \Quote\, f \equiv \QNf{f} $}
\DisplayProof

\\

\multicolumn{2}{c}{
\AxiomC{}
\UnaryInfC{$\vdash \Eval : \Pi (f : \Nat \rightarrow \Nat)\, (n : \Nat).\, {(\Quote\, f)} \CApp \Reif{n}{\Nat} \Red \Reif{f\ n}{\Nat} $}
\DisplayProof
}

\\

\multicolumn{2}{c}{

\AxiomC{$\vdash \Gamma$}
\AxiomC{$\vdash f : \Nat \rightarrow \Nat$}
\AxiomC{$\vdash n : \Nat$}
\AxiomC{$\vdash r : {(\Quote\, f)} \CApp \Reif{n}{\Nat} \Red \Reif{f\ n}{\Nat} $}
\QuaternaryInfC{$\Gamma \vdash \Eval\, f\, n \equiv r $}
\DisplayProof
}

\end{tabular}
\end{center}

\noindent where $\Reif{\cdot}{\Nat} \in \Trm{\cdot}{\Nat \rightarrow \Code}$ is defined internally by induction as
\begin{center}
\begin{tabular}{lcl}
$\Reif{\TZero}{\Nat}$ & $:=$ &$\CZero$ \\
$\Reif{\TSucc\, n}{\Nat}$ & $:=$ &$\CSucc\, \Reif{n}{\Nat}$
\end{tabular}
\end{center}
\noindent and for any context $\Gamma$ the external predicates $\PNf{\Gamma}$ and $\PNe{\Gamma}$ are mutually defined as
\begin{center}
\renewcommand{\arraystretch}{2}
\begin{tabular}{ccc}
\AxiomC{$x \in \Gamma$}
\UnaryInfC{$x \in \PNe{\Gamma}$}
\DisplayProof
&
\AxiomC{$n \in \PNe{\Gamma}$}
\AxiomC{$v \in \PNf{\Gamma}$}
\BinaryInfC{$n\, v \in \PNe{\Gamma}$}
\DisplayProof
&
\text{(recursors)}

\\

\AxiomC{$v \in \PNf{\Gamma}$}
\AxiomC{$v$ not closed}
\BinaryInfC{$\Quote\, v \in \PNe{\Gamma}$}
\DisplayProof

&

\multicolumn{2}{c}{
\AxiomC{$v \in \PNf{\Gamma}$}
\AxiomC{$v$ not closed}
\AxiomC{$w \in \PNf{\Gamma}$}
\AxiomC{$w$ not closed}
\QuaternaryInfC{$\Eval\, v\, w \in \PNe{\Gamma}$}
\DisplayProof
}

\\

\multicolumn{3}{c}{
\AxiomC{$n \in \PNe{\Gamma}$}
\UnaryInfC{$n \in \PNf{\Gamma}$}
\DisplayProof
}

\\

\AxiomC{\strut$A \in \PNf{\Gamma}$}
\AxiomC{$v \in \PNf{\Gamma, x : A}$}
\BinaryInfC{$\lambda x : A.\, v \in \PNf{\Gamma}$}
\DisplayProof

&

\AxiomC{\strut}
\UnaryInfC{$\TZero \in \PNf{\Gamma}$}
\DisplayProof

&

\AxiomC{\strut$v \in \PNf{\Gamma}$}
\UnaryInfC{$\TSucc\, v \in \PNf{\Gamma}$}
\DisplayProof

\\

\multicolumn{2}{c}{
\text{(type former are values)}
}

&

\text{(code and reduction formers are values)}

\end{tabular}
\end{center}

\noindent and $\QNf{\cdot}$ is a meta-function $\TNf{\cdot}{\Nat\rightarrow\Nat} \Rightarrow \TNf{\cdot}{\Code}$ satisfying
\[ \TEq{\cdot}{t}{u} \quad \text{implies}\quad \QNf{t} = \QNf{u} \]

\section{Turing-completeness}

In order to derive $\CT$ in $\CTTT$, we have to prove internally that there exists a computable partial function $\mathtt{run} : \Code \rightharpoonup \Nat$ s.t. whenever $t \Red \Reif{n}{\Nat}$ then $\mathtt{run}\, t \rhd n$. For this to make sense we probably need to assume that $\Code$ is a first-order datatype, but apart from that it should be possible to carry the proof in $\MLTT$ alone. % TODO Yannick

\section{Good metatheoretical properties}

We expect that $\CTTT$ is strongly normalizing and enjoys canonicity, and is thus consistent. Furthermore type-checking should be decidable. We show this exhibiting an NbE model, assuming some mild properties on the additional structures.

\begin{itemize}
 \item The meta-quote function $\QNf{\cdot} \in \TNf{\cdot}{\Nat\rightarrow\Nat} \Rightarrow \TNf{\cdot}{\Code}$ needs to pick a canonical representative in the equivalence class of convertible normal forms, so one has to care about stuff like $\eta$-expansion.
 \item The predicate $\Red$ should be canonical, i.e. for any two closed terms $t, u \in \Trm{\cdot}{\Code}$ then any two proofs $r, s \in \Trm{\cdot}{t \Red u}$ are convertible, that is we have a proof $\TEq{\cdot}{r}{s}$. This should hold using the intended model that $\Red$ is the inductive characterization of the unique accessibility graph of a deterministic, decidable one-step reduction.
 \item Conversion to a natural number should be enough to derive a proof of $\Red$.
 \item More stuff we realize on the way.
\end{itemize}

The model follows the now standard presentation of decidability of conversion via a logical relation by Abel.


\end{document}
